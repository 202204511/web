\documentclass[article,oneside,a4paper]{memoir}

\usepackage[utf8]{inputenc}
\usepackage[sc]{mathpazo}
\usepackage{xspace}
\usepackage{listings}
\usepackage{url}

\newcommand{\pulerau}{\texttt{pulerau}\xspace}

\begin{document}

\chapter{Guide til webgruppens arbejde}

\section{GitHub}

Udvikling af tutorgruppens hjemmeside og mailsystem fungerer via GitHub. Webfar
håndterer integration af kode i produktionen og stiller udviklingssites på
\pulerau til rådighed for de i webgruppen der ønsker det.

\section{Generel vejledning til git}

\paragraph{Identitet}
Det første, man gør, når man begynder at bruge Git, er at fortælle Git sit navn
og sin emailadresse. For eksempel:
\begin{lstlisting}
git config --global user.name 'Mathias Rav'
git config --global user.email 'rav@cs.au.dk'
\end{lstlisting}
Dette skal gøres på alle computere, hvorfra man udvikler, og det er vigtigt at
man altid bruger den samme identitet (dvs. navn og email).

\paragraph{Brug af Git}
Git er en del af webgruppens arbejde; \url{http://gitref.org/} er en glimrende
guide til at komme igang med brug af Git. Læs siden igennem fra start til slut.

\section{Vision}

Målet er en hjemmeside og et mailsystem med en overskuelig kodebase.
Det gamle mailsystem, som var i brug fra 2007 til 2012, var 2729 linjers
write-once read-never PHP.

Lamson er valgt til det nuværende mailsystem, fordi det skjuler alle de grumme
detaljer om SMTP.  Det betyder, at det er tilstrækkeligt med et par hundrede
linjers kode for at opfylde tutorgruppens emailbehov.

Django er et velunderstøttet web framework, og der er mange features, der kan
implementeres med få linjers kode. Vil du lave et RSS feed? Okay - du skal lave
en 40 linjers subclass af \texttt{Feed} og pege på den i din \texttt{urls.py}.
Den største udfordring er ofte at finde ud af præcis hvad man skal skrive.

\end{document}
