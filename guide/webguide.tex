\documentclass[article,oneside,a4paper]{memoir}

\usepackage[sc]{mathpazo}
\usepackage[final]{microtype}
\usepackage{url}
\usepackage[utf8]{inputenc}

\begin{document}

\title{Webgruppen}
\author{Mat/Fys-Tutorgruppen}
\date{19. februar 2013}
\maketitle

\chapter{Arbejdsplan}

\paragraph{RKFL$^2$}
Til dette fabelagtige arrangement d. 2. marts 2013 tager alle medlemmer af
gruppen en laptop med, så vi kan sætte udviklingsmiljøet op for alle.
(Webtumling har lavet en {\small README} med en kort instruktionsliste som alle
gennemgår på dagen under kyndig vejledning.)

Dernæst skal vi planlægge \emph{hackdays}, som er dage i løbet af 2013 (typisk
søndage) hvor vi mødes et sted på uni og udvikler hjemmesiden sammen.
I 2012 holdt webgruppen to af disse; målet i år er at holde 4-5 af disse inden
rusdagene da rushjemmesiden og bursystemet skal udvikles.
Når vi har lagt en plan for hackdays kan vi begynde at skitsere mere i detaljer
hvad for nogle features vi skal lave og hvem der kan tage ansvar for dem.

\paragraph{Hackdays}
De enkelte hackdays starter typisk en søndag kl. 12 og varer 3-5 timer.
De foregår typisk i et fancy mødelokale på datalogi (f.eks. Ada-333 eller Nygaard-327).
Enkelte medlemmer af webgruppen går sammen om at programmere. Det er rigtig hyggeligt.

\paragraph{Daglig vedligehold og udvikling}
Det påhviler den webansvarlige i tutorgruppens bestyrelse at fikse kritiske fejl på hjemmesiden,
og alle i webgruppen kan til hver en tid arbejde på hjemmesiden hvis de har tid og lyst.
Den obligatoriske del af gruppearbejdet består dog i de afholdte hackdays.

\chapter{Medlemmer}

Følgende tutorer er medlemmer af webgruppen i 2013.

\begin{itemize}\itemsep1pt \parskip0pt \parsep0pt
  \item Frederik Jerløv
  \item Jesper Madsen
  \item Jonas Termansen
  \item Mathias Dannesbo
  \item Mathias Rav (gruppeansvarlig)
  \item Philip Tchernavskij
  \item Søren Qvist Jensen
  \item Tobias Ansbak Louv
\end{itemize}

\chapter{Organisation}

Arbejdet organiseres gennem mails på listen \url{web@matfystutor.dk};
en fælles Google-kalender som minder os om datoer for hackdays; og en
GitHub-bruger hvor vi udveksler kode.

Både hjemmesiden og mailsystemet er open source, og kildekoden kan findes på
\url{www.github.com/matfystutor} hvor der også er en \emph{issue tracker}
som sporer feature requests og udviklingen af enkelte features.

Tutorhjemmesiden og mailsystemet hører hjemme på en ekstern server som Science
and Technology har stillet til rådighed for studenterforeningerne på
hovedområdet.  Den kaldes \texttt{pulerau} og drives primært af Steffen Videbæk
Petersen og Mathias Rav.

\section{Overblik over hjemmesidens kildekode}

Hjemmesidens kode består hovedsagligt af et \emph{Django project} kaldet
\texttt{mftutor}; en række \emph{Django apps}, heriblandt \texttt{tutor},
\texttt{events}, \texttt{news}, \texttt{activation}, \texttt{tutormail} og
\texttt{aliases}; og en masse web design i mapperne \texttt{tpl} og
\texttt{static}.
På dette tidspunkt rummer hjemmesiden lige under 2200 linjers Python kode.
%find -name transitiondb -prune -or -name '*.py' -exec wc {} +
%  2186   6132  76736 total
(Til sammenligning bestod den gamle tutorhjemmeside af 14000 linjers PHP,
%find web -name PDFClass -prune -or -name '*.php' -exec wc {} +
% 14422  47372 477157 total
den gamle rushjemmeside af 2500 linjers PHP
%find rus -name '*.php' -exec wc {} +
%  2514   8328  83843 total
og et gammelt adminsystem af 5000 linjers PHP.)
%find admin -name '*.php' -exec wc {} +
%  5020  17919 177159 total

Som en læsevejledning anbefales følgende filer, i den følgende rækkefølge:

\begin{enumerate}\itemsep1pt \parskip0pt \parsep0pt
  \item \texttt{mftutor/urls.py}
  \item \texttt{tutor/urls.py}
  \item \texttt{tutor/models.py}
  \item \texttt{models.py} i andre apps
  \item \texttt{urls.py} og \texttt{views.py} i alle apps
  \item \texttt{admin.py} i alle apps
\end{enumerate}
hvormed man kommer til at læse omtrent 1200 linjers kode.
%find \( -name urls.py -or -name models.py -or -name views.py -or -name admin.py \) -exec wc {} +
% 1278  3525 47801 total

\end{document}
